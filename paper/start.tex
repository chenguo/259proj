\begin{abstract}
        In an effort to increase instruction level parallelism, techniques have been developed such
as pipelining, out of order execution and operand forwarding.  However
these efforts come at a large power cost.  Out of order execution requires
a ReOrder Buffer (ROB), to commit the registers in program order.  This
structure has grown to be monolithic with numerous entries and read/write ports.
There are no current methods to efficiently and equally compare various
ROB advancements to improve power efficiency. We propose a loosely cycle
accurate model that keeps track of the power usages of the ROB.  Using the 
simulation tool, we are able to compare and contrast the affects of improvements
made on the ROB.  This also allows for a first line test to validate proof of
concepts for future power efficient ROB designs.
\end{abstract}

\terms{Power, Energy, Architecture}

\keywords{Power efficiency, ROB, Simulation} % NOT required for Proceedings

\section{Introduction}
Out of order execution is a common technique used to aggressively push instruction
throughput to meet the demands of modern computer users.  As the name suggests
this allows for instructions to be inflight while previous instructions may still be waiting 
for an available execution unit or operands.  To accomplish this, two additional structures/stages
are necessary  on top of the traditional pipelined architecture.  Both of these additional
structures are essentially memory elements. The first structure is to keep track of which instructions
are ready to be executed and the other, known as the reorder buffer (ROB), allows the processor to 
execute instructions out of order but commit the results of execution in program order to maintain 
program correctness.  

Throughout its evolution the the ROB has grown to a large monolithic structure that consumes
on average 7\% of the overall processor power budget\cite{rabaey}.  Various new implementations
of the ROB have been proposed in an effort to reduce power consumption.  Some of which 
dynamically change the the size of the ROB or propose a banked buffer structure and other proposals argues in the 
extreme case to eliminate read ports all together\cite{kucuk}\cite{kucuk2} \cite{kucuk3}.

With the rapid change of manufacturing technologies and other impacts on energy utilization it 
becomes hard to determine the gain of each new proposed implementation.  We propose a new
and loosely cycle accurate ROB simulator that provides a fair comparison of  different ROB designs.
ROB\_pwr is a simulator that tabulates each hardware action that would cost power.  This allows for a fair
comparison between each ROB design, without the concern for the underlying technology. 

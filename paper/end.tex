\def\topfraction{.9}
\def\floatpagefraction{.8}

\section{Results}
For each of the three modeled SPEC benchmarks, 100,000 instruction were generated, and processed with each of the reorder buffer types. Instruction generation is seeded, so for each particular benchmark, each type of reorder buffer processes the same exact instructions. We model at 256 entry reorder buffer, running with a 4 way superscalar processor having three integer functional units and one FP functional unit. The results are tabulated in tables \ref{tab:bench1}, \ref{tab:bench2}, and \ref{tab:bench3}.

\subsection{Distributed ROB}

Our results for the distributed ROB are largely invalid for two reasons. One, much of the savings in the distributed ROB comes from shorter bit-lines and less copmlex entry selection from the size reduction of each individual buffer, however this is not currently modeled in our reorder-buffer simulator. Next, our implementation is actually buggy, as is obvious from the results.

From the first instruction stream, heavy on floating point instructions, we see the distributed buffer measures some thiry thousand less cycles to complete than the circular buffer. In \cite{kucuk4}, the improvement for FP instructions over the base buffer did improve very slightly (we note that they were also modelling a processor with many more FP functional units than the ones we used), but to the degree we see is impossible. We monitored the buffer and indeed found corruption in the contents within the first 30 instructions, but ultimately were unable to find the source of the error.

For all instruction streams, we note that while the circular ROB powered 17,936 bits pe cycle (bit-cycles / cycles), the distributed ROB powered 27,692 bits per cycle. This is largely due to the presence of the central FIFO queue, which must keep a set of pointers to the location of actual entries in the ROB components of each functional unit. The extra cost of this is offset by shorter bit-lines and word-select lines, which as previously mentioned are not currently tabulated.


\subsection{Dynamic and Optimized Dynamic ROB}
In the FP heavy 173.applu benchmark, neither of the dynamic ROB's fared well. The IPC degradation was minimal, as the basic dynamic buffer took 1.8\% extra cycles to complete the instruction streem, and the optimized version did little better, at 1.6\% more cycles. However, looking at the bit-cycles measurement, we see that not much bits were saved per cycle. Dividing by the number of cycles, we see that while the circular buffer powered an average of 17,936 bits per cycle, the dynamic buffer powered 17,256 bits per cycle, and the optimized dynamic buffer powered 17,490 bits per cycle. All this is to indicate that the buffer very rarely emptied to a point where a parititon could be turned off. This is likely because we have only 1 FP unit, thus the buffer was never able to commit more than one instruction per cycle.

For the integer heavy instruction stream, the dynamic ROB's performed much better. The IPC degradation was again minimal, at 1.1\% for the dynamic ROB and amazingly only two cycles for the optimized version. This could be because the circular buffer itself rarely filled up, therefore even if the optmized dynamic buffer was running with a smaller capacity stalls due to the buffer being full was extremely rare. In terms of power saved, we see both the dynamic buffers were able to turn off partitions with regularity. Where the ciruclar buffer averaged 17,936 bits per cycle, in this instruction stream the dynamic ROB averaged 11,903 bits, and the optimized versino averaged 12,299. For the total bit-cycles, the dynamic buffer saved 33\%, while the optimized version saved 31\%. We see here that the optimized buffer opted to have more partitions enabled in exchange for better IPC performance.

Lastly, the mixed 183.equake instruction stream saw the buffers perform similarly to the integer set. The dynamic buffer suffered 6.3\% IPC degradation, while the optimized dynamic ROB had an IPC drop of 2.2\%. Bit cycles wise, the dynamic buffer saved 34\%, while the optimized dynamic buffer was able to save 36\%. In terms of bits enabled per cycle, the dynamic ROB 11,106 bits per cycle, and the optimized buffer enabled 11,233 bits per cycle.

For the FP heavy stream, we saw essentially no advantages using the dynamic approach, due to our modeled processor having only one FP functional unit clogging the buffer. With the other two streams, we saw dramatic improvement in power savings from the dynamic buffer, in the form of saved bit cycles. It is observed that the optimized ROB is a bit more performance minded than the regular dynamic ROB, averaging more bits per cycle and less cycles to complete. In the integer heavy set, this combination yielded a 2.2\% increase in bit-cycles for the optimized dynamic buffer, while in the mixed stream the difference in cycles pushed the optimized dynamic ROB's bit-cycles to be 2.8\% better than the non-optimized case.

\subsection{Omited Output Port ROB}
The Latch ROB results shown in the tables are produced from the parameters 
speficed above with 16 FIFO retention latches. Looking at the result tables,
 we see that the OOP ROB has comperable performance (in cycle) when compared 
to basline Circular ROB.  Showing that the additional slowdown caused by the
 non-availibilty of the operand when it enters the ROB is negligible, which
matches with the designers argument \cite{kucuk}.  

Next we see that the rest of the power usage parameters are have all been 
increased.  Both the bit cycles and transitions to high are slightly higher.
This is to be expected since there is the additional retention latches 
strucure, which makes the ROB larger and thus more bits to have a transition
to the high state.  The same logic applies to the large increase of the
register comparators used.  Each check for an operand has to check the
register tag of all the elements in the ROB, and the rentention latches 
as well.  

From the results tables, it appears that there is no reasonable reason to use
the OOP ROB.  This is untrue, as the the read port usages for the OOP drops
significantly.  Since we approximate about 6\% forwarding from the ROB, of 
100,000 instructions, we would need to foward about 6,000 operands (as a
lower bound, assuming all the instructions were INT ops).  This equates to
192,000, bit lines that needs to be driven for operand forwading.  Also with
the elimination of the forwarding ports, the design of the ROB becomes much 
simplier as well. With increasing forward rates required, the energy savings
will increase.


\section{Conclusions}
ending stuff
